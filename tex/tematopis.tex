\documentclass[11pt]{article}

\usepackage{sectsty}
\usepackage{graphicx}
\usepackage[T1]{fontenc}
\usepackage[polish]{babel}
\usepackage[utf8]{inputenc}

%% Margins
%\topmargin=-0.45in
%\evensidemargin=0in
%\oddsidemargin=0in
%\textwidth=6.5in
%\textheight=9.0in
%\headsep=0.25in


%%%% REMARKS %%%%%%%%%
\usepackage{color}
\definecolor{brickred}      {cmyk}{0   , 0.89, 0.94, 0.28}

\makeatletter \newcommand \kslistofremarks{\section*{Uwagi} \@starttoc{rks}}
  \newcommand\l@uwagas[2]
    {\par\noindent \textbf{#2:} %\parbox{10cm}
{#1}\par} \makeatother

\newcommand{\ksremark}[1]{%
{%\marginpar{\textdbend}
{\color{brickred}{[#1]}}}%
\addcontentsline{rks}{uwagas}{\protect{#1}}%
}
%%%%%%%%%%%%%% END OF REMARKS %%%%%%%%%%%

\frenchspacing

\title{Aplikacja mobilna do nauki znaków kanji}
\author{Aleksandra Kyc}
\date{\today}

\begin{document}
\maketitle	

% Optional TOC
% \tableofcontents
% \pagebreak

%--Paper--

\section*{Skrócony opis działania aplikacji}

\ksremark{Tu jeszcze trzeba napisać, co w ogóle jest aplikacja.}
Aplikacja będzie skierowana \ksremark{może: przeznaczona} na urządzenia mobilne. Użytkownik będzie miał możliwość przeglądania dostępnych znaków pogrupowanych po różnych poziomach zaawansowania. Oprócz predefiniowanych grup będzie dostępna możliwość tworzenia własnych zestawów do nauki, niezależnych od powyższego podziału. Każdy znak będzie posiadał czytania \textit{on'yomi} i \textit{kun'yomi}, znaczenie w języku angielskim oraz listę znaczących słów zawierających ten znak. Po wyborze danego znaku lub zestawu znaków użytkownik może rozpocząć naukę. W celu efektywnego nauczania aplikacja poprosi ucznia nie tylko znaczenie danego znaku, ale też o rozwiązanie zadań z różnymi złożeniami danego znaku z innymi. Aplikacja będzie wspierać nie tylko samouków, ale też naukę w grupie dzięki systemowi współdzielenia zestawów przez różnych użytkowników. Do grupy będzie można też dodać użytkownika o uprawnieniach nauczyciela, który oprócz tworzenia zestawów może oglądać wyniki uczniów w nauce, zadawać zadania domowe oraz testy sprawdzające umiejętności. Ostatnim rodzajem użytkownika będzie administrator, który ma możliwość zarządzania użytkownikami i ich uprawnieniami.

%--/Paper--

\end{document}
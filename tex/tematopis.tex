\documentclass[11pt]{article}

\usepackage{sectsty}
\usepackage{graphicx}
\usepackage[T1]{fontenc}
\usepackage[polish]{babel}
\usepackage[utf8]{inputenc}

% Margins
\topmargin=-0.45in
\evensidemargin=0in
\oddsidemargin=0in
\textwidth=6.5in
\textheight=9.0in
\headsep=0.25in

\title{ Aplikacja mobilna do nauki znaków kanji }
\author{ Aleksandra Kyc }
\date{\today}

\begin{document}
\maketitle	

% Optional TOC
% \tableofcontents
% \pagebreak

%--Paper--

\section*{Skrócony opis działania aplikacji}

Aplikacja b\k{e}dzie skierowana na urządzenia mobilne. Użytkownik b\k{e}dzie miał możliwość przeglądania dost\k{e}pnych znaków pogrupowanych po różnych poziomach zaawansowania. Oprócz predefiniowanych grup b\k{e}dzie dost\k{e}pna możliwość tworzenia własnych zestawów do nauki, niezależnych od powyższego podziału. Każdy znak b\k{e}dzie posiadał czytania \textit{on'yomi} i \textit{kun'yomi}, znaczenie w j\k{e}zyku angielskim oraz list\k{e} znaczących słów zawierających ten znak. Po wyborze danego znaku lub zestawu znaków użytkownik może rozpocz\k{a}ć nauk\k{e}. W celu efektywnego nauczania aplikacja poprosi ucznia nie tylko znaczenie danego znaku, ale też o rozwi\k{a}zanie zadań z różnymi złożeniami danego znaku z innymi. Aplikacja b\k{e}dzie wspierać nie tylko samouków, ale też nauk\k{e} w grupie dzi\k{e}ki systemowi wspóldzielenia zestawów przez różnych użytkowników. Do grupy b\k{e}dzie można też dodać użytkownika o uprawnieniach nauczyciela, który oprócz tworzenia zestawów może ogl\k{a}dać wyniki uczniów w nauce, zadawać zadania domowe oraz testy sprawdzaj\k{a}ce umiej\k{e}tności. Ostatnim rodzajem użytkownika b\k{e}dzie administrator, który ma możliwość zarz\k{a}dzania użytkownikami i ich uprawnieniami.

%--/Paper--

\end{document}